% смена стиля оформления страниц
% ничего не устанавливает как для верхнего, так и для нижнего колонтитула
\fancyhf{}
% установка верхнего колонтитула
\fancyhead[C]{\thepage}
\renewcommand{\headrulewidth}{0pt}
\pagestyle{fancy}

\fancypagestyle{plain}{%
	\fancyhf{}
	% убрать разделительную линию 

	\renewcommand{\headrulewidth}{0pt}
	\fancyhead[C]{\thepage}
}
\defaultfontfeatures{Mapping=tex-text,Scale=MatchLowercase}
\newfontfamily\cyrillicfont{Times New Roman}[Script=Cyrillic]
\newfontfamily\cyrillicfonttt{Times New Roman}[Script=Cyrillic]

% Настроика qзыкa
\setdefaultlanguage{russian}

% настройка полей
\geometry{top=2cm}
\geometry{bottom=2cm}
\geometry{left=3cm}
\geometry{right=1.5cm}
\geometry{bindingoffset=0cm}

%  использование полуторного интервала
\renewcommand{\baselinestretch}{1.5}


% настройка ссылок и метаданных документа
\hypersetup{unicode=true,colorlinks=true,linkcolor=black,citecolor=green,filecolor=magenta,urlcolor=cyan,
	pdftitle={\docname},
	pdfauthor={\studentname},
	pdfsubject={\subject},
	pdfcreator={\studentname},
	pdfproducer={LateX},
	pdfkeywords={\subject}
}

% настройка подсветки кода и окружения для листингов
\usemintedstyle{vs}
\newenvironment{code}{\captionsetup{type=listing}}{}


% оформления подписи рисунка
\captionsetup[figure]{labelsep = period}

% подпись таблицы
\DeclareCaptionFormat{hfillstart}{\hfill#1#2#3\par}
\captionsetup[table]{format=hfillstart,labelsep=newline,justification=centering,skip=10pt,textfont=bf}


% путь к каталогу с рисунками
\graphicspath{{Src/images/}}


\counterwithin{figure}{section}
\counterwithin{table}{section}

\titlelabel{\thetitle.\quad}

% перенос знаков в формулах (по Львовскому)
\newcommand*{\hm}[1]{#1\nobreak\discretionary{} {\hbox{$\mathsurround=0pt #1$}}{}}

\begin{luacode*}
	function split(inputstr, sep)
	if sep == nil then
	sep = "%s"
	end
	local t={}
	for str in string.gmatch(inputstr, "([^"..sep.."]+)") do
	table.insert(t, str)
	end
	return t
	end

	function boldenfirstword(line)
	local words = split(line, " ")
	local fword = words[1]
	table.remove(words, 1)
	tex.print("\\textbf{"..fword.."} "..table.concat(words," "))
	end
\end{luacode*}

% declare a wrapper in TeX
\newcommand{\boldenfirstword}[1]{\directlua{boldenfirstword("#1")}}


\newcommand{\sortitem}[1]{%
	\DTLnewrow{list}% Create a new entry
	\DTLnewdbentry{list}{description}{\boldenfirstword{#1}}% Add entry as description
}

\newenvironment{sortedlist}{%
	\DTLifdbexists{list}{\DTLcleardb{list}}{\DTLnewdb{list}}% Create new/discard old list
}{%
	\DTLsort{description}{list}% Sort list
	\begin{itemize}%
		\DTLforeach*{list}{\theDesc=description}{%
		\item[] \theDesc}% Print each item
	\end{itemize}%
}


\titleformat{\section}
{\normalfont\fontsize{14}{16}\bfseries}{\thesection}{1em}{\MakeUppercase}[]

\titleformat{name=\section,numberless}
{\normalfont\fontsize{14}{16}\bfseries\centering}{\llap}{1em}{\MakeUppercase}

\titleformat{\subsection}
{\normalfont\fontsize{12}{14}\bfseries}{\thesubsection}{1em}{}

\titleformat{\subsubsection}
{\normalfont\fontsize{12}{14}\bfseries}{\thesubsubsection}{1em}{}


% Делает заголовки в TOC ->  Uppercase 
\makeatletter   %tex.stackexchange.com/questions/129677/
% Always capitalize section headings in ToC (tex.stackexchange.com/a/156917)
\let\@oldcontentsline\contentsline
\def\contentsline#1#2{%
    \expandafter\ifx\csname l@#1\endcsname\l@section
        \expandafter\@firstoftwo
    \else
        \expandafter\@secondoftwo
    \fi
    {\@oldcontentsline{#1}{\MakeUppercase{#2}}}%
    {\@oldcontentsline{#1}{#2}}%
}
\makeatother

\usepackage{enumitem}
\setlist[enumerate]{label*=\arabic*.}
\AddToHook{cmd/section/before}{\clearpage}

\usepackage{tocloft}
\usepackage{float}
\renewcommand{\cftsecleader}{\cftdotfill{\cftdotsep}}

\usepackage{lipsum}
\usepackage{titlesec}
\renewcommand{\thesection}{\arabic{section}{}}
\renewcommand{\thesubsection}{\arabic{section}.\arabic{subsection}.{}}
\renewcommand{\thesubsubsection}{\arabic{section}.\arabic{subsection}.\arabic{subsubsection}.{}}

\usepackage{xstring}
\def\FixCaptionLabel#1{%
	\IfSubStr{#1}{.}{%
		\StrSubstitute[1]{#1}{.}{}}{#1}}



\usepackage{subcaption}

% \DeclareCaptionLabelFormat{custom}
% {%
% 	#1. \FixCaptionLabel{#2}
% }

% \DeclareCaptionLabelFormat{customappendix}
% {
% 	#2. \figurefix{#1}
% }
% % Separator style
% \DeclareCaptionLabelSeparator{custom}{ }

% % Caption format    
% \DeclareCaptionFormat{custom}
% {%
% 	#1#2\textbf{#3}
% }

% \DeclareCaptionFormat{customappendix}
% {%
% 	#2#1  \\ \textbf{#3}
% }


% \captionsetup
% {
% 	format=custom,%
% 	labelformat=custom,%
% 	labelsep=custom
% }



\newcommand\signature[2]{% Name; Department
	\noindent\begin{minipage}{5cm}
		\noindent\vspace{3cm}\par
		\noindent\rule{5cm}{1pt}\par
		\noindent\textbf{#1}\par
		\noindent#2%
	\end{minipage}}


\newcommand\conformationsignature[2]{
	\begin{table}[h]
		\begin{tabular}{p{7cm}}
			\\
		\end{tabular}
		\begin{tabular}{p{3cm}}
			\\\hline
			#1
		\end{tabular}
		\begin{tabular}{p{4cm}}
			\\\hline
			#2
		\end{tabular}
	\end{table}
}


\newcommand\insertdate[1][\today]{\vfill\begin{flushright}#1\end{flushright}}
\usepackage{csquotes}
\usepackage{plantuml}
\usepackage{xcolor}
\definecolor{light-gray}{gray}{0.95}
\newcommand{\codeword}[1]{\colorbox{light-gray}{\texttt{#1}}}
\newcommand{\cw}[1]{\colorbox{light-gray}{\texttt{#1}}}
\newcommand{\boldword}[1]{\textbf{#1}}

\lstset{
	breaklines=true,
	breakatwhitespace=false,
	xleftmargin=1em,
	%frame=single,
	%numbers=left,
	numbersep=5pt,
}


\newcommand\mylstcaption{}

\mdfdefinestyle{mymdstyle}{
	hidealllines=true,
	%middleextra={
	%  \node[anchor=west] at (O|-P)
	%    {\lstlistingname~\thelstlisting\  (Cont.):~\mylstcaption};},
	%secondextra={
	%  \node[anchor=west] at (O|-P)
	%    {\lstlistingname~\thelstlisting\  (Cont.):~\mylstcaption};},
	splittopskip=2\baselineskip
}

\surroundwithmdframed[style=mymdstyle]{lstlisting}
\newmdenv[style=mymdstyle]{mdlisting}

\newcounter{bibcount}

\makeatletter
\patchcmd{\@lbibitem}{\item[}{\item[\hfil\stepcounter{bibcount}{\thebibcount.}}{}{}
\setlength{\bibhang}{2\parindent}
\renewcommand\NAT@bibsetup%
[1]{\setlength{\leftmargin}{\bibhang}\setlength{\itemindent}{-\parindent}%
	\setlength{\itemsep}{\bibsep}\setlength{\parsep}{\z@}}
\makeatother
\bibliographystyle{agsm}

% Создание рамки и заднего фона для листингов
\BeforeBeginEnvironment{minted}{\begin{tcolorbox}}%
\AfterEndEnvironment{minted}{\end{tcolorbox}}%

% \usepackage[nameinlink]{cleveref}

\definecolor{Black}{rgb}{0,0,0}
