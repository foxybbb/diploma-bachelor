
\section*{\centering ЗАКЛЮЧЕНИЕ}

В ходе выполнения бакалаврской работы были проведен анализ мини роботов, а так же рассмотрены характеристики  мини роботов манипуляторов и их недостатки. Описана проблематика использования роботов совместно с людьми, необходимости особенностям создания короботов. 
Рассмотрены варианта двигателя, выбор и варианты управления BLDC мотором, а также вариант использования управление двигателями, был выбран метод векторного регулирования с обратной связью.

Разработаны функции и технические характеристики для разрабатываемой системы управления для коллаборативного мини робота манипулятора. Была рассмотрена и принята за основу иерархическая парадигма систем управления роботами для разделения принципиальных схем систему управления для мини робота, а также, были описаны взаимодействие систем тактического, стратегического и исполнительного управления. 
Алгоритмы разработаны основываясь на описанных иерархическим уровням архитектуры:
\begin{itemize}
	\item Алгоритм работы системы тактического управления; 
	\item Алгоритм работы системы исполнительного управления.
\end{itemize}
Была разработана функциональная схема устройства в соответствии с выбранными элементами.
\begin{itemize}
	\item Gimbal GBM4008H-150T, GM5208-120T, GM3506;
	\item Абсолютного Энкодера AS5600;
	\item Устройство стратегического контроля Raspberry Pi Zero W 2;
	\item Силовые ключи SIR680DP;
	\item Драйвера для силовых транзисторов L6385ED;
	\item Предадтчик шины данных TCAN1462DRQ1;
	\item DC-DC преобразователь АP64502QSP.
\end{itemize}
Были рассмотренные какие математические операции для решения задач связи с этим было проведено тестирование и анализ доступных микроконтроллеров под задачи векторного регулирования и кинематических задач. Разработаны две принципиальные схемы устройств, с необходимыми расчетами элементов для систем тактической и стратегических управления и для систем исполнительного управления.
\begin{itemize}
    \item Разработан алгоритм системы управления мини роботом манипулятором;
    \item Алгоритм управления системы тактического управления мини роботом;
    \item Алгоритм управления системы исполнительного управления мини роботом.
\end{itemize}
Таким образом можно сказать, что прошла успешно демонстрация возможностей использования двигателя типа от карданных камер (gimbal motors) для создания системы управления коллаборативного мини робота манипулятора с учетом упрощения их механической конструкции. Данная тема остается актуальная, появляться компоненты с похожими характеристиками, но с меньшим размером.
