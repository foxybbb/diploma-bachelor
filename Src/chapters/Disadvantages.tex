
The understanding of robots is something universal that can be applied to different tasks, in fact it is so that there are not so many special 
different variants in architecture among robot arms, the main differences lie in the different technical characteristics and technical implementation 
of these robots. It is possible to be overwhelmed by the sheer number of different robot models when studying them. In order to get acquainted with robotic
 manipulators, two models have been chosen that can be analysed in detail and from which a conclusion can be drawn. 

\subsection*{MyCobot}
Among the models that are not very expensive, \textbf{myCobot} works occupies an honourable place, this robot spreads in the low price segment, in this and stood interest 
what possibilities developers offer to the user of this device. , some of them can be considered really small in size and suitable for the requirements.
Certain disadvantages exist that are often found in existing models of robot manipulators.


% % Lua code to modify the label
% \begin{luacode}
%     function fixDots(input)
%         out = string.gsub(input, "%.", "", 1)
%         tex.print(out)
%     end
%     \end{luacode}
    
%     \newcommand{\fixdots}[1]{\directlua{fixDots("#1")}}
%     \crefformat{figure}{Figure ~\fixdots{#2#1#3}}
    


\begin{figure}[H]
	\centering
	\includegraphics[width=0.4\textwidth]{Src/images/myCobot.png}
	\caption{MyCobot 280}
    \label{fig:mycobot1}
    
\end{figure}
One of the popular models that is available on the market and can be available is the \textbf{myCobot} model robot, shown in the picture \ref*{fig:mycobot1}, as well as technical specifications of the robots represents in the table \ref*{tab:mycobot1}.


\begin{table}[H]
    \caption{Specifications MyCobot Robot Arm}\label{tab:mycobot1}
    \centering
    \begin{tabular}{|l|c|c|}
        \hline
        \textit{\textbf{Parameter ame}} & \multicolumn{1}{l|}{\textit{\textbf{Value}}} & \multicolumn{1}{l|}{\textit{\textbf{Units}}} \\ \hline
        DOF                  & 6     & -     \\ \hline
        Payload              & 0.25  & kg    \\ \hline
        Weight               & 0.8   & kg    \\ \hline
        PRepeatability       & ± 0.5 & mm    \\ \hline
        Power Voltage        & DC 12 & V     \\ \hline
        Power Current        & 5     & A     \\ \hline
        Gear type            & -     & Steel \\ \hline
        Shell                & -     & Metal \\ \hline
        Max. reach           & 280     & mm \\ \hline
        Cost                 & 800    & Euro \\ \hline
        \end{tabular}
 \end{table}

 


 In most cases, one of the main factors in this type of robot is the use of more low-cost actuators that drive the robot's axis. This may be the use of more low-cost gearboxes and motors that greatly reduce the cost of the entire robot, but the lifetime of these systems is relatively short.

 In the process of testing the basic specifications that was explained that with not all the parameters that are presented in the table \ref{mycobot1} data are not all reliable, the accuracy of movement is highly dependent on the load at the end of the 6 axis of the robot. Thus, the repeatability of the robot movement was not within \textbf{\pm 0.5 mm}, but could reach deviation around \textbf{ 5 mm}. As shown in figure \ref{mycobot2} 

 \begin{figure}[H]
	\centering
	\includegraphics[width=0.4\textwidth]{Src/images/mycobot2.png}
	\caption{Robot repeatability measurement example}
    \label{mycobot2}
\end{figure}

In fact, the servomotors used in hobbyist versions of robotic manipulators, such as the one shown in figure \ref{mycobot3}, are significantly less accurate and reliable than those used in their industrial counterparts. One of the main reasons for this is that these types of servos are typically designed with cost-effectiveness and basic functionality in mind, rather than high accuracy or durability. This can lead to inconsistent performance, frequent failures and a shorter overall life.


\begin{figure}[H]
	\centering
	\includegraphics[width=0.4\textwidth]{Src/images/mycobot3.png}
	\caption{Hobbyist Servo Motor}
    \label{mycobot3}
\end{figure}


In addition, the lightweight construction of these servomotors can make them more susceptible to physical impact and other environmental factors. This lack of robustness often results in frequent maintenance, which can lead to increased downtime and reduced productivity.

Hobbyist servomotors often have less sophisticated control systems, which may not provide the advanced features required for precise control in complex applications. 
Based on all of the above factors, we can conclude that this type of robot is not suitable for the task due to the lack of rigidity and accuracy of the design itself, as well as deficiencies in the actuators of each axis.

\subsection*{Kawasaki}

On the other side of the available robot arms are serious systems from reputable manufacturers, an example of such robots available for review is from the Japanese corporation Kawasaki - Robot Arm FS03N, the appearance can be seen in the picture {}, , according to the data sheet look at the table \ref{tab:Kawasaki1} you can see the main characteristics of the robot, this robot has better accuracy of movement, repeatability and weight of the travelling load, the main disadvantage of this robot is the redundancy of all the characteristics for tasks related to the movement of small objects. This choice of robot reduces the efficiency of technological production.


 \begin{figure}[H]
	\centering
	\includegraphics[width=0.6\textwidth]{Src/images/FS03N.jpg}
	\caption{Kawasaki FS03N}
    \label{ Kawasaki1}
\end{figure}



\begin{table}[H]
    \caption{Specifications Kawasaki Robot Arm FS03N}\label{tab:Kawasaki1}
    \centering
    \begin{tabular}{|l|c|c|}
        \hline
        \textit{\textbf{Parameter ame}} & \multicolumn{1}{l|}{\textit{\textbf{Value}}} & \multicolumn{1}{l|}{\textit{\textbf{Units}}} \\ \hline
        DOF                  & 6     & -     \\ \hline
        Payload              & 3 & kg    \\ \hline
        Weight               & 20   & kg    \\ \hline
        Repeatability        & ± 0.02 & mm    \\ \hline
        Power Voltage        & AC 230 & V     \\ \hline
        Max Power Current    & 9    & A     \\ \hline
        Gear type            & -     & Steel \\ \hline
        Shell                & -     & Aluminium \\ \hline
        Max. reach           & 620     & mm \\ \hline
        Cost                 & 12 000     & Euro \\ \hline
        \end{tabular}
 \end{table}

%  \subsection*{Meca500}
% \begin{figure}[H]
% 	\centering
% 	\includegraphics[width=0.4\textwidth]{Src/images/Meca500.jpg}
% 	\caption{Meca 500}
%     \label{Meca1}
% \end{figure}

% \begin{table}[H]
%     \caption{Specifications for Meca 500 Robot Arm}\label{tab:Meca1}
%     \centering
%     \begin{tabular}{|l|c|c|}
%         \hline
%         \textit{\textbf{Parameter ame}} & \multicolumn{1}{l|}{\textit{\textbf{Value}}} & \multicolumn{1}{l|}{\textit{\textbf{Units}}} \\ \hline
%         DOF                  & 6     & -     \\ \hline
%         Payload              & 0.5 & kg    \\ \hline
%         Weight               & 4.5   & kg    \\ \hline
%         PRepeatability       & ± 0.005 & mm    \\ \hline
%         Power Voltage        & DC 24 & V     \\ \hline
%         Max Power Current    & 9    & A     \\ \hline
%         Gear type            & -     & Steel \\ \hline
%         Shell                & -     & Aluminium \\ \hline
%         Max. reach           & 330     & mm \\ \hline
%         Cost                 & 20 000     & Euro \\ \hline
%         \end{tabular}
%  \end{table}