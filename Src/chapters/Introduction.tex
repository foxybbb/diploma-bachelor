

The active dissemination of robots after the mid-20th century led to the automation of nearly all aspects of production processes, including assembly, welding, painting, quality control, and packaging. This also had a significant impact on logistics, warehousing, and distribution, where robotic systems are used for sorting, moving, and storing goods. Advances in mechanics, electronics, and digital devices led to the creation of a variety of devices.

The use of existing universal mini manipulator robots is an effective and rational solution, due to their small size and parameters suitable for working with small parts and structures. However, despite the advantages of mini manipulator robots, they have disadvantages. The main disadvantage is the danger of robots working together with humans. To organize the work of these robots, it is necessary to create special enclosed work areas to minimize potential risk, and working together with a human is deemed impossible. A mini robot cannot become collaborative if power or force is limited for each link. To make the robot collaborative, changes need to be made to the robot's design. The size of mini manipulator robots creates limitations for all system elements, and installing torque sensors is not the best solution, as they take up additional space in each link of the robot and complicate the mini robot's design.

Collaboration is provided by a combination of information devices, motors, and control program algorithms. Among the listed set, motors play an important role. The most suitable type of electric motor is the brushless DC motor, among the many options of brushless DC motors, Gimbal motors are characterized by high power with a small size.

The goal of the work is: Development of a control system for a collaborative mini manipulator robot to demonstrate the capabilities of using Gimbal-type motors as the robot's actuating device.

The work contains 5 sections. The first section analyzes mini manipulator robots available on the market, various motor options are analyzed, and a choice of motor type is made, defining functions and technical parameters for the mini robot control system. The second section develops a structural diagram of the mini robot control system based on a hierarchical robot control paradigm, and an algorithm for its operation is created. The third section selects functional elements and analyzes microcontrollers for developing the control system, creates a functional diagram, and provides its description. The fourth section develops the principal circuit with its description and calculation of circuit elements. The fifth part develops control system algorithms and software implementation on a microcontroller.