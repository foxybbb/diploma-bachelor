

Прогресс в области механики, электроники и цифровых устройств, привел к созданию инновационных устройств. Это в свою очередь поспособствовало прогрессу, существенному повышению качества и возможностей ранее разработанных решений. Одним из примеров недавних разработок можно считать устройства группы gimbal. Устройство группы Gimbal и представляет собой сложную систему взаимодействия электродвигателей, электроники и датчиков для обеспечения плавности и стабилизации камер. Главной особенностью является использование бесколлекторных электрических двигателей постоянного ток. BLDC моторы обладают высокой эффективностью и мощностью при небольшом весе. 

В настоящее время роботы манипуляторы являются главным элементом процесса автоматизации производств. Обуславливается это гибкостью, точностью, скоростью и способностью выполнять поставленные задачи. Однако не во всех сферах роботы заменили ручной труд. Большинство работ связанных со сборкой деталей небольших размеров выполняются людьми. Для выполнения задач такого рода использование существующих универсальных индустриальных роботов является не эффективным и не рациональным решением, виду их большого размера и параметров несоотвествующих для работы с мелкими деталями и конструкциями.

Данная работа посвящена разработке системы управления для мини робота на основе двигателей от Gimbal. Актуальность разработки таких систем управления повышается в связи с тенденциями в необходимости разработок мини роботов и использование мини роботов в ближайшие годы.

Работа содержит 5 разделов. В первом разделе производится анализ устройств gimbal, мини роботов манипуляторов, которые доступны на рынке и их характеристики. Исследуются варианты управления двигателями gimbal и производится разработка функций и технических параметров для системы управления мини роботом.  Во втором разделе производится разработка структурной схемы на основе иерархической парадигмы управления роботом, а также алгоритма её описания для системы управления мини роботом. В третьем разделе производится разработка функциональной схемы и описание и назначения ее элементов. В четвертом разделе производится разработка принципиальной схемы с ее описанием, расчет элементов схемы. В пятой части производится разработка алгоритмов и программная реализация на микроконтроллере. 
