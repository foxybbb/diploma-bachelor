

Активное распространение роботов после середины 20 века привело к автоматизации почти всех аспектов производственных процессов, включая сборку, сварку, покраску, контроль качества и упаковку. Это также оказало значительное влияние на логистику, складское хозяйство
и дистрибуцию, где роботизированные системы используются для сортировки, перемещения и
хранения товаров. Прогресс в области механики, электроники и цифровых устройств, привел к
созданию инновационных устройств. 
Использование существующих универсальных мини роботов манипуляторов является эффективным и рациональным решением, ввиду их небольшого размера и параметров соответствующих для работы с мелкими деталями и конструкциями. Однако несмотря на преимущества мини роботов манипуляторов, они обладают недостатками. Главным недостатком является опасность работы роботов совместно с людьми. Для организации работы данных роботов необходимо создавать специальные закрытые рабочие зоны, для минимизации возможного риска, а работа совместно с человеком представляется невозможной.  Мини робот не может стать коллаборативным если ограничить мощность или силу для каждого звена. Для того что бы робот являлся коллаборативным необходимо производить изменения в конструкции самого робота. Размер мини роботов манипуляторов создает ограничения
для всех элементов системы, установка датчиков момента силы не является лучшим решением,
так как занимают дополнительное пространство в каждом звене робота и усложняют конструкцию мини робота. 

Наиболее подходящий тип электродвигателя является бесщеточный двигатель постоянного тока, для обеспечения всеми особенностями коллаборативного робота, тип BLDC двигателя Gimbal motors наиболее подходящим. 

Целью работы является: Демонстрация возможности использования двигателя типа от карданных камер (gimbal motors) для создания системы управления коллаборативного мини робота манипулятора с учетом упрощения их механической конструкции.

Работа содержит 5 разделов. В первом разделе производится анализ мини роботов манипуляторов, которые доступны на рынке и их характеристики. Исследуются различные варианты двигателей и производится выбор, разработка функций и технических параметров для системы управления мини роботом.  Во втором разделе производится разработка структурной схемы на основе иерархической парадигмы управления роботом, а также алгоритма её описания для системы управления мини роботом. В третьем разделе производится разработка функциональной схемы и описание и назначения ее элементов. В четвертом разделе производится разработка принципиальной схемы с ее описанием, расчет элементов схемы. В пятой части производится разработка алгоритмов и программная реализация на микроконтроллере. 
