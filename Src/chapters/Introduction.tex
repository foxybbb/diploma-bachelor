

Активное распространение роботов после середины 20 века привело к автоматизации почти всех аспектов производственных процессов, включая сборку, сварку, покраску, контроль качества и упаковку. Это также оказало значительное влияние на логистику, складское хозяйство
и дистрибуцию, где роботизированные системы используются для сортировки, перемещения и
хранения товаров. Прогресс в области механики, электроники и цифровых устройств, привел к
созданию всевозможных устройств. 

Использование существующих универсальных мини роботов манипуляторов является эффективным и рациональным решением, ввиду их небольшого размера и параметров соответствующих для работы с мелкими деталями и конструкциями. Однако несмотря на преимущества мини роботов манипуляторов, они обладают недостатками. Главным недостатком является опасность работы роботов совместно с людьми. Для организации работы данных роботов необходимо создавать специальные закрытые рабочие зоны, для минимизации возможного риска, а работа совместно с человеком представляется невозможной.  
Мини робот не может стать коллаборативным если ограничить мощность или силу для каждого звена. Для того что бы робот являлся коллаборативным необходимо производить изменения в конструкции самого робота. Размер мини роботов манипуляторов создает ограничения для всех элементов системы, установка датчиков момента силы не является лучшим решением, так как они занимают дополнительное пространство в каждом звене робота и усложняют конструкцию мини робота.

Коллаборативность обеспечивается совокупностью информационных устройств, двигателей и алгоритмами управляющих программ. Среди перечисленного набора, важную роль выполняет двигатели. Наиболее подходящий тип электродвигателя является бесщеточный двигатель постоянного тока, среди множества вариантов бесщёточным двигателей постоянного тока, двигатели Gimbal моторы характеризуются большой мощностью при малом размере.

Целью работы является: Разработка системы управления коллаборативного мини робота манипулятора для демонстрации возможностей использования двигателей типа Gimbal в качестве исполнительного устройства робота.
%Демонстрация возможности использования двигателя типа от карданных камер (gimbal motors) для создания системы управления коллаборативного мини робота манипулятора с учетом упрощения их механической конструкции.

Работа содержит 5 разделов. В первом разделе произведен анализ мини роботов манипуляторов, которые доступны на рынке. Анализируются различные варианты двигателей и произведен выбор типа двигателя, определение функций и технических параметров для системы управления мини роботом.  Во втором разделе разработана структурная схема системы управления мини роботом на основе иерархической парадигмы управления роботом, а также создан алгоритм её работы. В третьем разделе выполнен выбор функциональнальных элементов и произведен анализ микроконтроллеров для разработки системы управления, произведена функциональная схема и выполнено её описание. В четвертом разделе производится разработка принципиальной схемы с ее описанием и расчет элементов схемы. В пятой части  разработаны алгоритмы работы системы управления и программная реализация на микроконтроллере. 
