



The modern drive towards the development of automation in manufacturing
 processes and the growth of automated factories has led to an increasing 
 set of requirements for sorting systems and robotic process systems. 

 In addition, the desire to miniaturise the parts and components of each system is 
 forcing the development of new methods to improve the efficiency of the process
 , taking into account the size of the parts. To solve the problem, it is necessary 
 to choose the smallest solutions,Solutions must be simple, cheap and efficient components of 
 the complex.

 The problem of moving small parts is very important.
 In many areas, process improvements contribute to the reduction of system parts and components.
Therefore, it is necessary to manipulate small parts in the manufacturing process, to place them in the right places at the right angle. Unfortunately, a large number of small parts in production and assembly lines are handled manually these days.

For tasks that require automated movements, robots are great, but robot arms are more logical. Smaller robotic arms that can be placed on a table can be used, but they require a large amount of space. This increases the space required for the task, so it is not quite reasonable to use these robots. It is also worth considering the table robots were developed to work with large objects, for accurate movement and location of small objects is not always possible due to the existence of backlash and the complexity of turning the long axis at a small angle. To increase the efficiency and reduce the size of the manipulator robot, it is necessary to develop a minirobot that will meet the above parameters.
The development of any robot arm is a complex task that involves the interaction of complex systems. It is necessary to take into account the subtleties of design, hardware and software.Due to the complexity of the overall development of the theme of the bachelor's thesis will be concerned only with the control system of the mini robot manipulator itself.

The aim of the bachelor's thesis is a control system for a mini robot manipulator. 

In order to achieve the set goal of the bachelor's thesis, it is necessary to:

\begin{itemize}[itemsep=0pt]
    \item Determination of technical parameters, functions and modes of operation of an industrial minirobot
    \item Development of mechanical designs of industrial minirobot 
    \item Determination of technical parameters, functions and modes of operation of the minirobot control system
    \item Development of the structural scheme of the minirobot control system
    \item Development of functional and circuit diagrams of the minirobot control system
    \item Development of algorithms and control programmes for the minirobot control system
    \item Testing of the minirobot control system
\end{itemize}